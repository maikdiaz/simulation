
\documentclass[12pt,letterpaper]{article}
\usepackage[utf8]{inputenc}
\usepackage[spanish, es-tabla]{babel}
\usepackage[version=3]{mhchem}
\usepackage[journal=jacs]{chemstyle}
\usepackage{amsmath}
\usepackage{amsfonts}
\usepackage{amssymb}
\usepackage{makeidx}
\usepackage{xcolor}
\usepackage[stable]{footmisc}
\usepackage[section]{placeins}
%Paquetes necesarios para tablas
\usepackage{longtable}
\usepackage{array}
\usepackage{xtab}
\usepackage{multirow}
\usepackage{colortab}
%Paquete para el manejo de las unidades
\usepackage{siunitx}
\sisetup{mode=text, output-decimal-marker = {,}, per-mode = symbol, qualifier-mode = phrase, qualifier-phrase = { de }, list-units = brackets, range-units = brackets, range-phrase = --}
\DeclareSIUnit[number-unit-product = \;] \atmosphere{atm}
\DeclareSIUnit[number-unit-product = \;] \pound{lb}
\DeclareSIUnit[number-unit-product = \;] \inch{"}
\DeclareSIUnit[number-unit-product = \;] \foot{ft}
\DeclareSIUnit[number-unit-product = \;] \yard{yd}
\DeclareSIUnit[number-unit-product = \;] \mile{mi}
\DeclareSIUnit[number-unit-product = \;] \pint{pt}
\DeclareSIUnit[number-unit-product = \;] \quart{qt}
\DeclareSIUnit[number-unit-product = \;] \flounce{fl-oz}
\DeclareSIUnit[number-unit-product = \;] \ounce{oz}
\DeclareSIUnit[number-unit-product = \;] \degreeFahrenheit{\SIUnitSymbolDegree F}
\DeclareSIUnit[number-unit-product = \;] \degreeRankine{\SIUnitSymbolDegree R}
\DeclareSIUnit[number-unit-product = \;] \usgallon{galón}
\DeclareSIUnit[number-unit-product = \;] \uma{uma}
\DeclareSIUnit[number-unit-product = \;] \ppm{ppm}
\DeclareSIUnit[number-unit-product = \;] \eqg{eq-g}
\DeclareSIUnit[number-unit-product = \;] \normal{\eqg\per\liter\of{solución}}
\DeclareSIUnit[number-unit-product = \;] \molal{\mole\per\kilo\gram\of{solvente}}
\usepackage{cancel}
%Paquetes necesarios para imágenes, pies de página, etc.
\usepackage{graphicx}
\usepackage{lmodern}
\usepackage{fancyhdr}
\usepackage[left=4cm,right=2cm,top=3cm,bottom=3cm]{geometry}

%Instrucción para evitar la indentación
%\setlength\parindent{0pt}
%Paquete para incluir la bibliografía
\usepackage[backend=bibtex,style=chem-acs,biblabel=dot]{biblatex}
\addbibresource{references.bib}

%Formato del título de las secciones

\usepackage{titlesec}
\usepackage{enumitem}
\titleformat*{\section}{\bfseries\large}
\titleformat*{\subsection}{\bfseries\normalsize}

%Creación del ambiente anexos
\usepackage{float}
\floatstyle{plaintop}
\newfloat{anexo}{thp}{anx}
\floatname{anexo}{Anexo}
\restylefloat{anexo}
\restylefloat{figure}

%Modificación del formato de los captions
\usepackage[margin=10pt,labelfont=bf]{caption}

%Paquete para incluir comentarios
\usepackage{todonotes}

%Paquete para incluir hipervínculos
\usepackage[colorlinks=true, 
            linkcolor = blue,
            urlcolor  = blue,
            citecolor = black,
            anchorcolor = blue]{hyperref}

%%%%%%%%%%%%%%%%%%%%%%
%Inicio del documento%
%%%%%%%%%%%%%%%%%%%%%%

\begin{document}
\renewcommand{\labelitemi}{$\checkmark$}

\renewcommand{\CancelColor}{\color{red}}

\newcolumntype{L}[1]{>{\raggedright\let\newline\\\arraybackslash}m{#1}}

\newcolumntype{C}[1]{>{\centering\let\newline\\\arraybackslash}m{#1}}

\newcolumntype{R}[1]{>{\raggedleft\let\newline\\\arraybackslash}m{#1}}

\begin{center}
	\textbf{\LARGE{tutorial \#1 de NetLogo}}\\
	\vspace{7mm}
	\textbf{\large{Est: Michael Santiago Diaz}}\\
	\vspace{4mm}
	\textbf{\large{Simulación computacional}}\\
	\textbf{\large{Universidad de los llanos}}\\
	\textbf{\large{Profesor: Ph.D Ángel Cruz}}\\
	\today
\end{center}

\vspace{10mm}

\section*{\centering Resumen}

En este documento se presentará el desarrollo del primer tutorial de simulación en NetLogo, el cual tiene como objetivo mostrar de manera breve la experiencia adquirida atrevez del seguimiento del tutorial, respondiendo las preguntas contenidas en éste con base a las modificaciones que se pide aplicar, y cual es su efecto sobre el modelo o sobre la misma herramienta, de tal forma que se mostrará evidencia de la realización con las algunas capturas de pantalla.

\section{Introducción}

NetLogo es un entorno programable de modelado  para simular fenómenos naturales y sociales. Es especialmente adecuado para modelar sistemas complejos  que se desarrollan en el tiempo. Los modeladores pueden dar instrucciones a cientos o miles de "agentes" independientes todos operando en paralelo. Esto hace que sea posible explorar la relación entre el nivel micro del comportamiento de los individuos y los patrones a nivel macro que emergen de la interacción de muchos individuos.
 El modelo sobre el cual se realizó este primer tutorial es un modelo clásico de depredador - presa denominado \textit{Wolf sheep  predation} ubicado en la biblioteca de modelos de la herramienta.
 
\section{Desarrollo}

Una vez localicé en donde se encontraba a el modelo en la ruta \textit{Archivo $>$ Biblioteca de modelos} $>$ \textit {simple models} $>$ \textit{ biology } $>$ \textit{Wolf Sheep Predation} (\textit{ ver figura 1}) surgieron dos ventanas, una de ellas (\textbf{Controles}) con botones, deslizadores y monitores que permiten modificar las condiciones del modelo, y un cuadro de gráfica (Populations); la otra (Vista) con un rectángulo de color negro hasta los bordes laterales con otro rectángulo sin color de relleno y con borde de color blanco.\textit{Ver figura 2.}

\begin{figure}[h!]
\begin{floatrow}
\centering
\caption{Indicaión de ubicación del moldelo}
\includegraphics[width=15cm]{./imagenes/ubicacion_modelo_1_.jpg}
\label{fig:ruta del modelo}
\floatfoot{\small{FUENTE: Autor}}
\end{floatrow}
\end{figure}

\begin{figure}[h!]
\begin{floatrow}
\centering
\caption{Ventanas que se abren con el modelo}
\includegraphics[width=15cm]{./imagenes/modelo_abierto_2_.jpg}
\label{fig:controles y vista}
\floatfoot{\small{FUENTE: Autor}}
\end{floatrow}
\end{figure}


\subsection{¿Qué le aparece en la vista?}

Al presionar el botón “setup” de la ventana de controles, el rectángulo con borde blanco de la ventana de vista es rellenado de un fondo verde con unas pintas blancas que al parecer representan a las ovejas y negras que representan a los lobos, pero la porción sobre la que está situado continua en negro.

\begin{figure}[h!]
\begin{floatrow}
\centering
\caption{Apariencia de la vista}
\includegraphics[width=10cm]{./imagenes/ventana_Vista_3_.jpg}
\label{fig: vista}
\floatfoot{\small{ \hspace{65pt} FUENTE: Autor}}
% * <cristian.carranza.homes@unillanos.edu.co> 2015-11-07T19:08:10.953Z:
%
% ^.
\end{floatrow}
\end{figure}

\vspace{2cm}
\subsection{¿Qué le está sucediendo a las poblaciones de lobos y ovejas a medida que está corriendo el modelo? ¿Alguna vez obtendrá resultados diferentes si ejecuta el modelo en repetidas ocasiones manteniendo la misma configuración?}

Después de correr el modelo varias veces durante un tiempo no muy prolongado (alrededor de 320 tiks) la población de lobos disminuye hasta llegar a cero, por lo que “no hay nada” que haga disminuir la población de ovejas y ésta sigue aumentando infinitamente. A largo plazo el resultado siempre es el mismo.\textit{Ver la figura 4}.
\vspace{5cm}

\begin{figure}[h!]
\begin{floatrow}
\centering
\caption{Comportamiento de la población de ovejas a largo plazo}
\includegraphics[width=10cm]{./imagenes/Vista_controles_4_.jpg}
\label{fig: comportamiento con mas de 300 ticks}
\floatfoot{\small{ \hspace{65pt} FUENTE: Autor}}
\end{floatrow}
\end{figure}

\subsection{¿Qué pasó con las ovejas a través del tiempo?}

Con la configuración por defecto con la que está inicialmente el modelo, en aproximadamente 100 ticks la población de las ovejas empieza a disminuir poco después de alcanzar un punto máximo. El comportamiento de la población de ovejas la representa la línea azul mientras que a la población de lobos la representa la línea roja.\textit{Ver la figura 5}.

\begin{figure}[h!]
\begin{floatrow}
\centering
\caption{Comportamiento de la población de ovejas a 100 ticks aproximadamente}
\includegraphics[width=10cm]{./imagenes/botones_grafica_5_.jpg}
\label{fig: vista}
\floatfoot{\small{ \hspace{65pt} FUENTE: Autor}}
\end{floatrow}
\end{figure}
\vspace{5cm}

\subsection{¿Qué le hizo este switch al modelo? ¿Fue el mismo resultado de la ejecución previa?}

Al activar el switch de la hierba \textbf{(grass)} y correr el modelo presionando el botón setup y luego el botón go, se ha añadido una tercera línea en que representa el comportamiento de la hierba en relación con el comportamiento de las otras poblaciones. Ésta tercera variable incluida en el modelo afecta de manera importante el comportamiento de las poblaciones de lobos y ovejas ya que al dejar correr el modelo en la misma cantidad de ticks aproximadamente éstas disminuyeron considerablemente respecto ejecuciones anteriores del modelo sin activar el switch de hierba. El resultado es distinto al de la ejecución anterior.\textit{Ver figura 6}.

\begin{figure}[h!]
\begin{floatrow}
\centering
\caption{Comportamiento del modelo con la tercera variable grass.}
\includegraphics[width=10cm]{./imagenes/botones_grafica_hierba_6_.jpg}
\label{fig: vista}
\floatfoot{\small{ \hspace{65pt} FUENTE: Autor.}}
\end{floatrow}
\end{figure}

\subsection{¿Qué sucedería con la población de ovejas si hay al comienzo de la simulación más ovejas  y menos lobos?}

Habiendo visto el modelo de presa-depresador de Lotka Volterra en un curso de ecuaciones diferenciales(antes de concer esta herramienta), puedo afirmar que la poblacion inicial no tiene mucho impacto respecto al comportamiento del modelo en general, siempre y cuado la poblacion inicial de ambas sea mayor que cero. Caso contrario al impacto que genera variar las tasas de crecimiento y los parametros que representan la inteacción entre poblaciones.

\subsection{¿Qué le ocurrió a la población de ovejas?}
Con el switch de la hierba apagado, habiendo disminuido la población inicial de lobos  y dejando ejecutar el modelo durante aproximadamente 100 ticks, el cambio mas evidente es la cantidad que alcanza la población de ovejas antes de empezar a declinar, ésta cantidad es mayor respecto al modelo con la configuración por defecto.\textit{Ver la figura 7}.

\begin{figure}[h!]
\begin{floatrow}
\centering
\caption{Comportamiento del modelo con menos población inicial de lobos}
\includegraphics[width=10cm]{./imagenes/menos_ovejas_iniciales_7_.jpg}
\label{fig: vista}
\floatfoot{\small{ \hspace{65pt} FUENTE: Autor}}
\end{floatrow}
\end{figure}

\subsection{¿Le sorprendió este resultado?, ¿Qué otros sliders o switches se pueden ajustar para ayudarle a la población de ovejas?}

No me sorprendió el resultado, otras controles que se pueden ajustar para ayudarle a la población de ovejas son los slider que representan la ganancia de energía que se genera al alimentarse y la tasa de reproducción de ambas especies.

\subsection{¿Qué le pasó a los lobos en esta ejecución?}
Habiendo ajustado el slider de reproducción de ovejas hasta dejarlo en 10\%, es decir, al doble de la tasa de los lobos; y además ajustando el numero inicial de cada población de ovejas y lobos a 80 y 50  respectivamente, en la ejecución del modelo con estos parámetros mencionados la población de lobos se mantiene creciente, en una escala menor a la de las ovejas pero aun así creciente.\textit{Ver la figura 8}.

\begin{figure}[h!]
\begin{floatrow}
\centering
\caption{Comportamiento del modelo aumentando el procentaje de producción de ovejas.}
\includegraphics[width=10cm]{./imagenes/mas_produccion_ovejas_8_.jpg}
\label{fig: vista}
\floatfoot{\small{ \hspace{65pt} FUENTE: Autor}}
\end{floatrow}
\end{figure}

\vspace{1cm}

\subsection{Control de la  ventana vista}
A medida de que mueve el deslizador de velocidad en esta ventana hacia la izquierda se refresca por cada tick la vista de las pintas blancas y negras que representan a las ovejas y lobos respectivamente, lo cual hace que la ejecución  del modelo vaya más lenta pero permite observar mejor su comportamiento. Diferente a lo que ocurre cuando el slider va a la izquierda, cuando va a la derecha o cuando esta desmarcado el check box de actualizar la vista, la vista se refresca cada que cierto número de ticks o puede no actualizarse sino hasta que ocurra algún evento de mouse que comprometa modificar el tamaño de la ventana, provocando que la ejecución del modelo vaya más rápido. \textit{ Obsérvese en la figura 9 el número ticks trascurridos}.
\begin{figure}[h!]
\begin{floatrow}
\centering
\caption{Comportamiento del modelo variando la velocidad de ejecución.}
\includegraphics[width=10cm]{./imagenes/control_velocidad_9_.jpg}
\label{fig: vista}
\floatfoot{\small{ \hspace{65pt} FUENTE: Autor}}
\end{floatrow}
\end{figure}

\subsection{¿Cuáles son los ajustes actuales para max-pxcor, pxcor-min, Max-pycor, min-pycor, y  patch size  (tamaño del parche)?}

\begin{longtable}{C{2cm}C{2cm}C{2cm}C{2cm}C{2cm}}
\caption{Configuracion de la vista.}
\label{tab:indicadorespH}\\
\hline
\bfseries{Max pxcor} & \bfseries{Min pxcor} & \bfseries{Max pycor} & \bfseries{Min pycor} & \bfseries{Patch size}\\
\hline
\endfirsthead
\hline
\endhead
\endfoot
\hline
\endlastfoot

\multirow{1}{*}{25} &-25 &25 &-25 &9  \\
\end{longtable}

\begin{figure}[h!]
\begin{floatrow}
\centering
\caption{Tamaño por defecto del patche}
\includegraphics[width=10cm]{./imagenes/tamano_patche_10.jpg}
\label{fig:tubes}
\floatfoot{\small{\hspace{65pt} FUENTE: Autor}}
\end{floatrow}
\end{figure}
\vspace{5cm}
\subsection{¿Qué números cambiaron? ¿Qué números no cambiaron?}

Al manipular (mover, estirar, acercar)  la perspectiva gráfica de la parcela (Rectángulo donde se pintan las ovejas y lobos) no se altera ni uno de sus ajustes.
\vspace{1cm}
\subsection{¿A cuántas baldosas de distancia está la baldosa (0,0) respecto a lado derecho de la habitación? ¿A Cuántas baldosas de distancia está la baldosa (0,0) respecto al lado izquierdo de la habitación?}

Para ambos casos, izquierda y derecha se encuentra a 12 baldosas de distancia.
\vspace{1cm}

\subsection{¿Qué le ocurrió a la forma de la vista?}

Al modificar los ajustes de la vista max-pxcor a 30 y el valor de max-pycor a 10 la vista tomó una forma de rectángulo acostado o cuyos lados más largos son el inferior y el superior.\textit{Ver la figura 11}.

\begin{figure}[h!]
\begin{floatrow}
\centering
\caption{Tamaño por defecto del modificado}
\includegraphics[width=10cm]{./imagenes/tamano_patche_modificado_11_.jpg}
\label{fig: Tamaño modificado del patche.}
\floatfoot{\small{\hspace{65pt} FUENTE: Autor}}
\end{floatrow}
\end{figure}

\end{document}
